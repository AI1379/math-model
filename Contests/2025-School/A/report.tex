\documentclass{article}

\usepackage[UTF8]{ctex}
\usepackage[
  a4paper,
  left=2.5cm,
  right=2.5cm,
  top=2.5cm,
  bottom=2.5cm,
  headsep=0.5cm,
  footskip=1cm
]{geometry}
\usepackage{titling}
\usepackage{amsmath}
\usepackage{amsfonts}
\usepackage{amssymb}
\usepackage{amsthm}
\usepackage{minted}

\renewcommand{\maketitlehookb}{
  \ifx\theauthor\empty
  \vspace{-3em}
  \else
  \begin{center}
    \textbf{\theauthor}
  \end{center}
  \fi
}

\title{分布式能源配电网风险分析}
\author{}
\date{\today}

\begin{document}
\maketitle

\begin{abstract}\label{sec:abstract}
  摘要

  \textbf{关键词:} 关键词1,关键词2,关键词3
\end{abstract}

\newpage

\section{问题重述}\label{sec:problem}

\subsection{问题背景}\label{sec:background}

随着全球能源转型和我国"双碳"目标的深入推进,可再生分布式能源(如光伏、风电)在配电网中的渗透率显著提升。
这种趋势不仅推动了清洁能源的利用,还对传统配电网的运行方式提出了新的挑战。
传统配电网设计以集中式发电和单向潮流为主,难以适应分布式能源出力波动性和不确定性带来的复杂运行场景。
分布式能源的接入改变了配电网的潮流分布,可能导致失负荷(因故障导致供电中断)或过负荷(线路电流超额定载流量10\%以上),从而增加系统运行风险。
此外,联经开关的存在为故障后功率转移提供了可能,但其操作复杂性进一步增加了风险评估的难度。
在实际应用中,不同类型用户(如工业、商业、政府机构)对供电中断的敏感度差异显著,需综合考虑经济损失和社会影响来量化风险。
现有研究多聚焦于分布式能源的接入技术或单一风险因素,缺乏对失负荷与过负荷风险的系统性建模与综合评估。
本研究基于62节点有源配电网系统,结合分布式能源的运行特性,构建风险评估模型,为配电网的规划与优化提供理论支持。
这种研究不仅在理论上丰富了配电网风险管理的知识体系,还在实践中为提升电网可靠性和经济性提供了科学指导,具有重要的现实意义。

\subsection{问题提出}\label{sec:problem_proposed}

本研究旨在通过数学建模量化分布式能源接入配电网后引发的失负荷与过负荷风险,为配电网的稳定运行提供系统化解决方案。
研究围绕以下核心任务展开:首先,需建立分布式能源接入配电网后失负荷与过负荷风险的数学模型,明确风险的概率分布及其危害程度,结合联经开关的功率转移机制优化风险计算。
然后,基于62节点有源配电网系统,应用所建模型分析分布式能源接入对系统风险的动态影响,揭示风险演变的规律。
这些任务通过理论建模、数据分析与仿真验证相结合,旨在为配电网运行提供科学的决策依据,提升其在高渗透率分布式能源场景下的可靠性和经济性。

\section{问题分析}\label{sec:analysis}

分布式能源接入配电网的风险评估涉及失负荷与过负荷的概率建模、风险演变分析、光伏容量影响评估及储能优化等多个子问题,呈现出高度的复杂性和综合性。
问题的核心在于分布式能源出力的波动性与不确定性对配电网潮流分布的深刻影响,这不仅改变了传统配电网的运行模式,还引入了新的风险来源。
失负荷风险源于故障导致的供电中断,其危害程度与用户类型和停电持续时间密切相关;过负荷风险则由线路电流超载引发,可能导致设备损坏或系统崩溃。
联经开关的存在为功率转移提供了灵活性,但其操作需考虑网络拓扑的动态变化,增加了建模难度。

\subsection{模型假设}\label{subsec:assumption}

为了简化模型分析,本文在建模过程中做出以下假设:

\begin{enumerate}
  \item 假设分布式能源出力为固定值,且不随时间变化。
  \item 假设配电网中所有节点的负荷均为已知值,且不随时间变化。
  \item 假设配电网的拓扑结构在分析期间保持不变,且联络开关变化状态离散。
  \item 假设配电系统故障发生独立,且故障率固定。
\end{enumerate}

\subsection{符号说明}\label{subsec:notation}

\section{模型建立与求解}\label{sec:model}

\subsection{问题 1 建模与求解}\label{subsec:problem1}

\subsubsection{问题 1 求解思路}\label{subsubsec:problem1_idea}

\subsubsection{问题 1 建模}\label{subsubsec:problem1_model}

\subsubsection{问题 1 求解}\label{subsubsec:problem1_solve}

\subsection{问题 2 建模与求解}\label{subsec:problem2}

\subsubsection{问题 2 求解思路}\label{subsubsec:problem2_idea}

\subsubsection{问题 2 建模}\label{subsubsec:problem2_model}

\subsubsection{问题 2 求解}\label{subsubsec:problem2_solve}

\section{模型分析检验}\label{sec:analysis_check}

\section{模型评价与推广}\label{sec:evaluation}

\section{附录}\label{sec:appendix}

\subsection{代码实现}\label{subsec:code}

% minted 环境构建速度较慢,因此暂时注释掉
% \inputminted{python}{Q1_risk_model.py}

\end{document}
